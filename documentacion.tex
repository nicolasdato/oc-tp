\documentclass[12pt,twocolum,a4paper]{article}
\usepackage[utf8]{inputenc}
\usepackage[spanish]{babel}
\usepackage{hyperref}

\title{Documentaci\'on del proyecto de \emph{Organizaci\'on de Computadoras} - \emph{Ta-Te-Ti}}
\author{Nicol\'as Dato y Nicol\'as Medina}
\date{03/11/2019}
\begin{document}

\maketitle

\newpage
\tableofcontents
\newpage

\section{Definiciones y especificaci\'on de requerimientos}

\subsection{Definici\'on general del proyecto de software}
	Este proyecto se basa en la creaci\'on del famoso juego \emph{Ta-Te-Ti}, el cual cuenta con 3 modos de juego, juagador vs jugador, jugador vs computadora y computadora vs computadora.

	El modo jugador vs jugador consite simplemente en turnos donde cada jugador realizara una jugada y se notificar\'a si esta se realiz\'o exitosamente, en caso de que no fuera as\'i (como intentar ubicar una ficha en una parte del tablero en el cual ya hab\'ia una) se pedir\'a realizar la jugada nuevamente. Las jugadas se ejecutaran decidiendo la posici\'on en la grilla, es decir, decidiendo el lugar mediante cordenadas $x$ e $y$ comenzando a contar desde $0$.

	Si se desea ejecutar uno de los dos modos donde participa la computadora, cuando sea su turno se ejecutar\'a el algoritmo de {\bf busqueda adversaria MIN-MAX} donde se buscar\'a la mejor jugara, dadas las caracter\'isticas del \emph{Ta-Te-Ti} esto significa que la computadora nunca pierde, s\'olo gana o empata.

    Cualquier persona que sepa las reglas del juego puede ser capaz de jugar, lo unico que tendr\'a que hacer es indicar $x$ e $y$ para colocar la ficha en la posici\'on que se quiera.

\subsection{Especificaci\'on de requerimientos del proyecto}
    La implementacion del proyecto se ha realizado en el lenguaje C, el modo de visualizaci\'on del men\'u y tablero se ha realizado en la consola. Se ha implementado los {\itshape TDA LISTA} y {\itshape TDA ARBOL} los cuales se encontrara en una libreria dinamica denominada {\bf libliar} y esta se utilizara para el algoritomo de {\bf busqueda adversaria MIN-MAX} el cual es implementado en un {\itshape TDA IA} . Tambi\'en se hizo uso de un {\itshape TDA PARTIDA} para el manejo del tablero, turnos, etc. Por ultimo, el programa principal se encargara de de manejar tanto el {\itshape TDA PARTIDA} como el {\itshape TDA IA} e ir\'a mostrando el estado actual del tablero.

    El usuario al iniciar el programa tendr\'a la opci\'on de iniciar una nueva partida o de salir del sistema, en caso de iniciar una nueva partida tendr\'a que elejir en qu\'e modo es el que desea jugar y luego decidir los turnos, tambi\'en se pedir\'a ingresar el nombre de los jugadores.

    El proyecto es un desarrollo original cumpliendo las pautas de implementacion dada por la catedra de la materia {\itshape organizaci\'on de computadoras}, a\'un as\'i dado el desarrollo de clases y de estructura de datos aprendido en materias anteriores los {\itshape TDA} implementados en el lenguaje C pueden ser reutilizados en proyectos/desarrollos posteriores.

\subsection{Especificaci\'on de los procedimientos}

\subsubsection{Procedimientos de desarrollo}
    Como se dijo anteriormente, la implementaci\'on se realiz\'o en el lenguaje C y se ha usado como IDE y compilador de dicho lenguaje el dado por la catedra llamado {\bf Code::Blocks y MinGW}. Se han usado varias librerias de este lenguaje, las primeras dos son {\itshape stdlib.h} y {\itshape stdio.h} las cuales son las librerias primordiales para el manejo de memoria y de I/O. Una de las otras librerias utilizadas fue {\itshape time.h} utilizada en el caso de que el usuario decida que empieza un jugador aleatorio. Otra libreria utilizada fue la de {\itshape string.h} usada para copiar cadena de caracteres de una manera mas sencilla.

    Para abarcar este proyecto se decidi\'o primero la implementaci\'on del {\itshape TDA LISTA} para luego poder implementar el {\itshape TDA ARBOL} y as\'i estar en condiciones de encarar el algoritmo de {\bf busqueda adversaria MIN-MAX}. Aun as\'i primero decidimos completar el {\itshape TDA PARTIDA} antes de desarrollar {\itshape TDA IA}.

    Una vez con todos los {\itshape TDA} completados se pas\'o a la implementaci\'on del programa principal, donde se implementa un menu facilitando la experiencia del usuario.

    Ya con todo finalizado se implement\' la libreria {\bf libliar} con los {\itshape TDA LISTA} y {\itshape TDA ARBOL}.

\subsubsection{Procedimientos de compilaci\'on y ejecuci\'on}
    Para el uso del mismo, el usuario, el cual es as\'i mismo un desarrollador, contar\'a con el archivo comprimido {\bf .zip} conteniendo todos los archivos .cpb, .h y .c para la compilac\'on y ejecuci\'on del programa.
		
		Dado que se genera una librer\'ia (\emph{libliar}), se requiere abrir 2 proyectos con \emph{Code::Blocks}, tp1.cbp y libliar.cbp. En las propiedades del proyecto tp1 se puede configurar que tp1 tiene de dependencia el proyecto libliar\footnote{Lamentablemente no se logr\'o hacer que esta configuraci\'on quede guardada para que al abrir el proyecto ya est\'e la dependencia agregada}, para que al compilar tp1 tambi\'en se compile libliar.
		
		Para compilar el proyecto, primero que hay compilar libliar, que va a generar los archivos \emph{bin\textbackslash Debug\textbackslash libliar.a} y \emph{bin\textbackslash Debug\textbackslash libliar.dll}.
		
		Con libliar compilado, se puede compilar el proyecto tp1, el cual va a linkearse con \emph{bin\textbackslash Debug\textbackslash libliar.a}\footnote{Con \emph{Code::Blocks} no se logr\'o utilizar el archivo .dll, la \'unica opci\'on que hab\'ia era linkear contra el archivo .a, de todas formas el archivo .dll se genera y est\'a disponible.}
		
		Una vez compilado el proyecto tp1, en \emph{bin\textbackslash Debug\textbackslash} se genera el archivo \emph{tp1.exe} para ser ejecutado.

\section{Arquitectura del sistema}

\subsection{Descripci\'on jer\'arquica}
    La arquitectura del sistema cuenta con un programa principal llamado {\bf main} el cual se encarga de usar el {\itshape TDA PARTIDA} para la creaci\'on y administraci\'on de la misma, tambi\'en en el {\bf main} usar\'a el {\itshape TDA IA} en casos de ser necesario, es decir, si el usuario desea iniciar uno de los dos modos donde juega la computadora.

    El {\itshape TDA IA} utiliza el algoritmo de {\bf busqueda adversaria MIN-MAX}, tambi\'en usara al {\itshape TDA PARTIDA} y {\itshape TDA ARBOL}.

    Por ultimo el {\itshape TDA ARBOL} usar\'a al {\itshape TDA LISTA} para guardar una lista de sus hijos.

\subsection{Diagrama de m\'odulos}

\subsection{Descripci\'on general de los m\'odulos}
\subsubsection{Descripci\'on general y prop\'osito}
\begin{itemize}
    \item {\bf TDA LISTA}: Implementa nodos enlazados los cuales apuntan al suguiente nodo y almacenan un elemento gen\'erico. La posici\'on es indirecta y tiene un nodo centinela.
	\item {\bf TDA ARBOL}: Implementa \emph{tnodos} los cuales apuntan a su \emph{tnodo} padre (en caso de no ser la raiz), almacenan un elemento generico y una lista de hijos.
    \item {\bf TDA PARTIDA}: Implementa una partida con informaci\'on actual de la misma. La partida guarda tanto los nombres de los jugadores, el turno del jugador al que le toca jugar, el estado altual de la partida (si alguien gan\'o, etc) y el modo de la partida que se est\'a jugando. El tablero apunta a una grilla de 3x3 la cual guarda el estado del mismo teniendo en cada posici\'on los valores $0$, $1$ o $2$ para el caso de una casilla vac\'ia, una ficha del jugador 1 o una ficha del jugador 2, respectivamente.
    \item {\bf TDA IA}: Implementa un estado y una busqueda adversaria. El estado guarda una grilla de 3x3 con la que fue llamada y la utilidad el cual representa el resultado de la grilla (si se sigue jugando, alguien gan\'o, etc). Busqueda adversaria apunta a un arbol y a dos enteros, el jugador max y el jugador min. Con esta informaci\'on el algoritmo buscar\'a la mejor jugada que puede realizar el jugador.
    \item {\bf main}: Implementa un men\'u en la consola el cual administra el {\itshape TDA PARTIDA} y el {\itshape TDA IA} en caso de ser necesario.
\end{itemize}

\subsubsection{Responsabilidad y restricciones}
\begin{itemize}
    \item {\bf TDA LISTA}: La lista espera que cuando se le ingrese un elemento ya tenga un espacio en memoria asignado. La lista, al eliminar o destrir espera una funci\'on con la cual eleminar\'a al elemento y liberar\'a el espacio asignado a memoria del mismo. Al insertar un elemento lo asignar\'a como siguiente a la posici\'on recibida.
    \item {\bf TDA ARBOL}: El arbol espera despu\'es de ser creado que se le asigne una raiz como primer elemento, y en caso de que se le intente asignar un nodo que no sea ra\'iz se informar\'a de un error. Tambi\'en cuando se desee insertar un elemento se le tiene que asignar un espacio en memoria al mismo previamente. Al eliminar o destruir, de igual manera que la lista se espera que se le pase una funci\'on por parametro la cual se encargara del eleminar y borrar de la memoria al elemento. Al insertar un elemento se esperan dos tnodos, el primero de ellos debe ser el padre al cual se le desea agregar un hijo y el segundo es el hermano derecho al nodo a insertar, en caso de que el segundo nose sea NULL el elemento se asigna como el ultimo hijo del tnodo padre.
    \item {\bf TDA PARTIDA}: La partida lo que espera como modo y turno son enteros, asociados a contantes ya definidas en el TDA, en el caso del nombre se espera como maximo 49 caracteres. Si dado al caso se intenta hacer un movimiento en una casilla ya ocupada, retornara un error.
    \item {\bf TDA IA}: La ia al momento de crear la busqueda adversaria lo que espera es que se le haya pasado correctamente la partida en su estado actual, para as\'i poder empezar con el algoritmo. En el caso de resultado esperado lo que espera la ia es que el resultado buscado siempre sea el mejor, es decir, gana max, ya que la ia nunca pierde, s\'olo gana o empata. Tambien espera dos punteros a enteros los cuales modificara con la posici\'on en la cual esta la mejor jugada posible.
    \item {\bf main}: El main tiene limitaciones al ejecutarse por consola, las cuales son que los nombres de los jugadores no podran exceder los 49 caracteres, y la segunda es que espara que cada vez que da una opci\'on a elejir espera un n\'umero entero.
\end{itemize}

\subsubsection{Dependencias}
\begin{itemize}
    \item {\bf TDA LISTA}: Este TDA no requiere de ningun paquete externo o librer\'ia ademas de las librerias esenciales proporcionadas por C.
    \item {\bf TDA ARBOL}: Este TDA de la librer\'ia del {\itshape TDA LISTA} ya mencionado.
    \item {\bf TDA PARTIDA}: Este TDA adem\'as de las librer\'ias esenciales de C requiere de las  librer\'ias {\itshape time.h} y {\itshape string.h}.
    \item {\bf TDA IA}: Este TDA requiere de los {\itshape TDA PARTIDA} y {\itshape TDA ARBOL} ya mencionados.
    \item {\bf main}: El main solo requiere del {\itshape TDA IA} debido a que este ya tiene contenido todo el resto de los TDA y librer\'ias.
\end{itemize}

\subsubsection{Implementaci\'on}
\begin{itemize}
    \item {\bf TDA LISTA}: Este TDA se encuentra implementado en la librer\'ia din\'amina llamada {\bf libliar}.
    \item {\bf TDA ARBOL}: Este TDA se encuentra implementado en la librer\'ia din\'amica llamada {\bf libliar}.
    \item {\bf TDA PARTIDA}: Este TDA se encontrara definido en un .h denominado {\itshape partida.h} e implementado en un .c llamado {\itshape partida.c}. Estos archivos se encontraran en el {\bf .zip}.
    \item {\bf TDA IA}: Este TDA al igual que el anterior se encuentra definido en {\itshape ia.h} e implementado en {\itshape ia.c}. Estos archivos se encontrar\'an en el {\bf .zip}.
    \item {\bf main}: Este se encuentra implementado en {\itshape main.c}. Este archivo se encontrar\'a en el {\bf .zip}.
    \item {\bf libliar}: Librer\'ia din\'amica con los {\itshape TDA LISTA} y {\itshape TDA ARBOL}, se encontrar\'a en el {\bf .zip}.
\end{itemize}

\subsection{Dependencias externas}
\begin{itemize}
    \item {\bf stdio.h}: Es la librer\'ia que contiene las definiciones de las macros, las constantes, las declaraciones de funciones de la biblioteca est\'andar del lenguaje de programaci\'on C para hacer operaciones de entrada y salida, as\'i como la definici\'on de tipos necesarias para dichas operaciones.
    \item {\bf stdlib.h}: Es la librer\'a cabecera de la biblioteca est\'andar de prop\'osito general del lenguaje de programaci\'on C. Contiene los prototipos de funciones de C para gesti\'on de memoria din\'amica, control de procesos y otros.
    \item {\bf time.h}: Es la librer\'ia relacionada con formato de hora y fecha, es un archivo de cabecera de la biblioteca est\'andar del lenguaje de programaci\'on C que contiene funciones para manipular y formatear la fecha y hora del sistema.
    \item {\bf math.h}: Es una librer\'ia de cabecera de la biblioteca est\'andar del lenguaje de programaci\'on C dise\~nado para operaciones matem\'aticas b\'asicas. Muchas de sus incluyen el uso de n\'umeros en punto flotante.
    \item {\bf string.h}: Es na libreria de la biblioteca est\'andar del lenguaje de programaci\'on C que contiene la definici\'on de macros, constantes, funciones y tipos y algunas operaciones de manipulaci\'on de memoria relacionadas a cadena de caracteres.
\end{itemize}

\section{Descripcion de procesos y servicios ofrecidos por el sistema}

\end{document}

