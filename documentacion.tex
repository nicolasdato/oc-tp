\documentclass[12pt,twocolum,a4paper]{article}
\usepackage[utf8]{inputenc}
\title{Documentaci\'on}

\author{Nicol\'as Dato y Nicol\'as Medina }
\date{}
\begin{document}

\maketitle

\newpage
\tableofcontents
\newpage

\section{Definiciones y especificaci\'on de \newline requerimientos}

\subsection{Definici\'on general del proyecto de software}
\begin{frame}
    Este proyecto se basa en la creaci\'on del famoso juego Ta-Te-Ti, el cual cuenta con 3 modos de juego, juagador vs jugador, jugador vs computadora y computadora vs computadora.
    \newline
    El modo jugador vs jugador consitira simplemente en turnos donde cada jugador realizara una jugada y se notificara si esta se realizo exitosamente, en caso de que no fuera asi, (Intento ubicar una ficha en una parte del tablero en el cual ya habia una) se notificara de un error y podro realisar su jugada nuevamente. Las jugadas se ejecutaran decidiendo la pocision en la grilla, es decir, decidiendo el lugar mediante cordenadas $x$ e $y$.
    \newline
    Si se desea ejecutar uno de los dos modos donde participa la computadora, en este, la misma se ejecutara el algoritmo de {\bf busqueda adversaria \newline MIN-MAX}, donde se buscara la mejor jugara, es decir, la computaroa nunca pierde, gana o empata.
    \newline
    Cualquier persona consiente de las reglas del juego puede ser capaz, lo unico que tendra que hacer es marcar en $x$ e $y$ la posici\'on a la cual quieren insertar una ficha.
\end{frame}

\subsection{Especificaci\'on de requerimientos del proyecto}
\begin{frame}
    La implementacion del proyecto se ha realisado en el lengiaje C, el modo de visualizacion del menu y tablero se ha realizado en la consola. Se ha implementado los {\itshape TDA Lista} y {\itshape TDA Arbol} los cuales se encontrara en una libreria dinamica denominada {\bf libliar} y esta se utilizara para el algoritomo de {\bf busqueda adversaria MIN-MAX} el cual es implementado en un {\itshape TDA IA} . Tambien se hiso uso de un {\itshape TDA Partida} para el manejo del tablero, turnos, etc. Por ultimo, el programa principal se encargara de de manegar tanto el {\itshape TDA PARTIDA} como el {\itshape TDA IA} e ira mostrando el estado actual del tablero como el estado de la partica.
    \newline
    El usuario al iniciar el programa tendra la opcion de iniciar una nueva partida o de salir del sistema, en caso de inicuar una nueva partida tendra que elejir en que modo es el que desea jugar y luego decidir los turnos, tambien en caso de que alguno de los jugadores sea una persona podra/n insertar su nombre si haci lo desean.
    \newline
    El proyecto forma parte completamente de un desarrollo original cumpliendo las pautas de implementacion dada por la catedra de la materia {\itshape organizaci\'on de computadoras}, pero, aun asi dado el desarrollo de clases y de estructura de datos aprendido en materias anterios los {\itshape TDA} implementados en el lenguaje C puedes ser reutilizados en proyectos/desarrollos posteriores.
\end{frame}

\subsection{Especificaci\'on de los procedimientos}

\subsubsection{Procedimientos de desarrollo}
\begin{frame}
    Como se dijo antes la implementacion ha sid en el lenguaje C y se ha usado como programa compilador de dicho lenguaje el dado por la catedra llamado {\bf Code::Blocks}. Se has usado varias librerias de este lenguaje, las primeras dos son {\itshape stdlib.h} y {\itshape stdio.h} las cuales son las librerias primordiales en este lenguaje. Una de las otras librerias utilizadas fue {\itshape time.h} utilizada en el caso de que el usuario decida que empieza un jugador aleatorio. Otra libreria utilizada fue la de {\itshape string.h} usada para copiar cadena de caracteres de una manera mas sencilla .Y la otra libreria utilizada fue {\itshape math.h} utilizada para tener acceso a la funcion de valor absoluto.
    \newline
    Para abarcar este proyecto se decidio primero la implementacion del {\itshape TDA LISTA} para luego poder implementar el {\itshape TDA ARBOL} y asi estar en condiciones de encarar el algoritmo de {\bf busqueda adversaria MIN-MAX}. Aun asi primero decidimos completar el {\itshape TDA PARTIDA} antes de comensar el {\itshape TDA IA}. 
    \newline
    Una ves con todos los {\itshape TDA} completados se paso a la emplementacion del programa principal, donde se implementa un menu facilitando la experiencia del usuario.
    \newline
    Ya con todo finalizado se implemento la libreria dinamica {\bf libliar} con los {\itshape TDA LISTA} y {\itshape ARBOL}.
\end{frame}

\subsubsection{Procedimientos de instalaci\'on y prueba}
\begin{frame}
    Para el uso del mismo, el usuario, el cual es asi mismo un desarrollador, contara con el archivo comprimido {\bf .zip} el cual contara con todos los archovos .h y .c para la ejecucion del programa. El usuario lo unico que tendra que hacer es ejecutar el {\bf main} en su compilador de C y al hacerlo aparecera en consola un menu donde se podra jugar al Ta-Te-Ti. 
\end{frame}

\section{Arquitectura del sistema}

\subsection{Descripci\'on jer\'arquica}
\begin{frame}
    La arquitectura del sistema cuenta con un programa principal llamado {\bf main} el cual se encarga de usar el {\itshape TDA PARTIDA} para la creacion y administraci\'on de la misma, tambien en el {\bf main} usara el {\itshape TDA IA} en casos de ser necesario, es decir, si el usuario desea iniciar uno de los dos modos donde juega la computadora.
    \newline
    El {\itshape TDA IA} el usar el algoritmo de{\bf busqueda adversaria MIN-MAX} usara al {\itshape TDA PARTIDA}, asi mismi en la ejecucion del algorito tambien usara al {\itshape TDA ARBOL}.
    \newline
    Por ultimo el {\itshape TDA ARBOL} usara al {\itshape TDA LISTA} para guardar una lista de sus hijos.
\end{frame}

\subsection{Diagrama de m\'odulos}

\subsection{Descripci\'on general de los m\'odulos}
\subsubsection{Descripci\'on general y prop\'osito}
\begin{itemize}
    \item {\bf TDA LISTA}: Implemeta nodos enlazados los cuales apuntan al suguiente nodo y almacenan un elemento generico.
    \item {\bf TDA ARBOL}: Implementa tnodos los cuales apuntan a su tnodo padre (en caso de no ser la raiz), almacenan un elemento generico y una lista de hijos
    \item {\bf TDA PARTIDA}: Implementa una partida con informaci\'on actual de la misma. La partida, guarda tanto, los nombres de los jugadores de misma, el turno del jugador al que le toca jugar, el estado altual de la partida (Si alguien gano, etc) y el modo de la partida que se esta jugando. El tablero apunta a una grilla de 3x3 la cual guarda el estado del mismo. 
    \item {\bf TDA IA}: Implementa un estado y una busqueda adversaria. El estado guarda una grilla de 3x3 con la que fue llamada y la utilidad el cual representa el resultado de la grilla (Si se sigue jugando, alguien gano, etc). Busqueda adversaria apunta a un arbol y a dos enteros, el jugador max y el jugador min.
    \item {\bf main}: Implementa un menu en la consolo el cual administra el {\itshape TDA PARTIDA} y el {\itshape TDA IA} en caso de ser necesario.
\end{itemize}

\subsubsection{Responsabilidad y restricciones}
\begin{itemize}
    \item {\bf TDA LISTA}: La lista espera que cuando se le ingrese un elemento ya tenga un espacio en memoria ya asignado. La lista, al eleminar o destrir espera una funci\'on con la cual eleminara al elemento y liberara el espacio asignado a memoria del mismo. Al insertar un elemento lo asignara siguiente a la posici\'on recibida.
    \item {\bf TDA ARBOL}: El arbol espera despues de ser creado, espera que se le asigne una raiz como primer elemento, y en caso de que se le intente asignar una cuando esta ya esta creada se informara de un error. Tambien cuando se desee insertar un elemento se le tiene que asignar un espacio en memoria al mismo previamente. Al eliminar o destruir, de igual manera que la lista, espera que se le pase una funci\'on por parametro la cual se encargara del eleminar y borrar de la memoria al elemento. Al insertar un elemento se esperan dos tnodos, el primero de ellos debe ser el padre al cual se le desea agregar un hijo y el segundo es el hermano derecho al nodo a insertar, en caso de que el segundo nose sea NULL el elemento se asigna como el ultimo hijo del tnodo padre.
    \item {\bf TDA PARTIDA}: La partida lo que espera como modo y turno son enteros, asociados a contantes ya definidas en el TDA, en el caso del nombre se espera como maximo 49 caracteres. Si dado al caso se intenta hacer un movimiento en una casilla ya ocupada, retornara un error.
    \item {\bf TDA IA}: La ia al momento de crear la busqueda adversaria lo que espera es que se le halla pasado correctamente la partida en su estado actual, para asi poder empezar con el algoritmo. En el caso de resultado esperado lo que espera la ia es que el resultado buscado siempre sea el mejor, es decir, gana max, ya que, la ia nunca pierde, gana o empata. Tambien espera dos punteros a enteros los cuales modificara con la posici\'on en la cual esta la mejor jugada posible.
    \item {\bf main}: El main tiene como limitaciones al ejecutarse por consola, las cuales son que los nombres de los jugadores no podran exceder los 49 caracteres, y la segunda es que espara que cada ves que da una opci\'on a elejir espera un numero entero.
\end{itemize}

\subsubsection{Dependencias}
\begin{itemize}
    \item {\bf TDA LISTA}: Este TDA no requiere de ningun paquete externo o libreria ademas de las librerias esenciales proporcionadas por C.
    \item {\bf TDA ARBOL}: Este TDA de la libreria del {\itshape TDA LISTA} ya mencionado.
    \item {\bf TDA PARTIDA}: Este TDA ademas de las librerias esenciales de C requiere de las  librerias {\itshape math.h}, {\itshape time.h} y {\itshape string.h}.
    \item {\bf TDA IA}: Este TDA requiere de los {\itshape TDA PARTIDA} y {\itshape TDA ARBOL} ya mencionados.
    \item {\bf main}: El main solo requiere del {\itshape TDA IA} debido a que este ya tiene contenido todo el resto de los TDA y librerias.
\end{itemize}

\subsubsection{Implementaci\'on}
\begin{itemize}
    \item {\bf TDA LISTA}: Este TDA se encuentra implementado en la libreria dinamina llamada {\bf libliar}.
    \item {\bf TDA ARBOL}: Este TDA se encuentra implementado en la libreria dinamica llamada {\bf libliar}.
    \item {\bf TDA PARTIDA}: Este TDA se encontrara definido en un .h denominado {\itshape partida.h} e implementado en un .c llamado {\itshape partida.c}. Estos archivos se encontraran en el {\bf .zip}.
    \item {\bf TDA IA}: Este TDA al igual que el anterior se encuentra definido en {\itshape ia.h} e implementado en {\itshape ia.c}. Estos archivos se encontraran en el {\bf .zip}.
    \item {\bf main}: Este se encuentra implementado en {\itshape main.c}. Este archivo se encontrara en el {\bf .zip}.
    \item {\bf libliar}: Libreria dinamica con los {\itshape TDA LISTA} y {\itshape TDA ARBOL}, se encontrara en el {\bf .zip}.
\end{itemize}

\subsection{Dependencias externas}
\begin{itemize}
    \item {\bf stdio.h}: Es la libreria que contiene las definiciones de las macros, las constantes, las declaraciones de funciones de la biblioteca estándar del lenguaje de programación C para hacer operaciones, estándar, de entrada y salida, así como la definición de tipos necesarias para dichas operaciones.
    \item {\bf stdlib.h}: Es la libreria cabecera de la biblioteca estándar de propósito general del lenguaje de programación C. Contiene los prototipos de funciones de C para gestión de memoria dinámica, control de procesos y otras.
    \item {\bf time.h}: Es la libreria relacionada con formato de hora y fecha, es un archivo de cabecera de la biblioteca estándar del lenguaje de programación C que contiene funciones para manipular y formatear la fecha y hora del sistema.
    \item {\bf math.h}: Es uan libreria de cabecera de la biblioteca estándar del lenguaje de programación C diseñado para operaciones matemáticas básicas. Muchas de sus incluyen el uso de números en punto flotante.
    \item {\bf string.h}: ES na libreria de la biblioteca estándar del lenguaje de programación C que contiene la definición de macros, constantes, funciones y tipos y algunas operaciones de manipulación de memoria relacionadas a la manipulacion de cadena de caracteres.
\end{itemize}

\section{Descripcion de procesos y servicios \newline ofrecidos por el sistema}

\end{document}

